\documentclass[twocolumn]{src/aastex631}
\newcommand{\vdag}{(v)^\dagger}
\newcommand\aastex{AAS\TeX}
\newcommand\latex{La\TeX}

\usepackage{amsmath}
\usepackage{cancel}

\shorttitle{AASTeX v6.31 Sample article}
\shortauthors{Schwarz et al.}
\watermark{DRAFT}
\graphicspath{{./}{figures/}{src/figures}}

\begin{document}

\title{Updated High-Temperature Opacities for The Dartmouth Stellar Evolution
Program and their Effect on the Jao Gap Location}

\correspondingauthor{Thomas M. Boudreaux}
\email{thomas.m.boudreaux.gr@dartmouth.edu, thomas@boudreauxmail.com}

\author[0000-0002-2600-7513]{Thomas M. Boudreaux}
\affiliation{Department of Physics and Astronomy, Dartmouth College, Hanover, NH 03755, USA}

\author[0000-0003-3096-4161]{Brian C. Chaboyer}
\affiliation{Department of Physics and Astronomy, Dartmouth College, Hanover, NH 03755, USA}


\begin{abstract}

	The Jao Gap, a 17 percent decrease in stellar density at M$_{G} \sim$ 10
	identified in both Gaia DR2 and EDR3 data, presents a new method to probe
	the interior structure of stars near the fully convective transition mass.
	The Gap is believed to originate from convective kissing instability
	wherein asymmetric production of He$^{3}$ causes the core convective zone
	of a star to periodically expand and contract and consequently the stars’
	luminosity to vary. Modeling of the Gap has revealed a sensitivity in its
	magnitude to a population’s metallicity primarily through opacity. Thus
	far, models of the Jao Gap have relied on OPAL high-temperature radiative
	opacities. Here we present updated synthetic population models tracing the
	Gap location modeled with the Dartmouth stellar evolution code using the
	OPLIB high-temperature radiative opacities. Use of these updated opacities
	changes the predicted location of the Jao Gap by $\sim$0.05 mag as compared
	to models which use the OPAL opacities.

\end{abstract}

\keywords{Stellar Evolution (1599) --- Stellar Evolutionary Models (2046)}

\section{INTRODUCTION}\label{sec:intro}
\citet{Jao2018} discovered a novel feature in the Gaia $G_{BP}-G_{RP}$
color-magnitude-diagram. Around $M_{G}=10$ there is an approximately 17\%
decrease in stellar density of the sample of stars \citeauthor{Jao2018}
considered. Subsequently, this has become known as either the Jao Gap, or Gaia
M dwarf Gap. Section \ref{sec:p1} will go into more detail regarding the
physics underpinning this feature; however, in brief convective instabilities
in the core are believed to form for stars straddling the fully convective
transition mass (0.3 - 0.35 M$_{\odot}$) \citep{Baraffe2018}. These
instabilities interupt the normal, slow, main sequence luminosity evolution of
a star and resulting in lower than expected luminosities {\color{red} [WORDING]} \citep{Jao2020}.

The Jao Gap, inherently a feature of M dwarf populations, provides an enticing
and unique view into the interior physics of these stars \citep{Feiden2021}.
This is especially important as, unlike more massive stars, M dwarf seismology
is currently infeasible due to the short periods and extremely small
magnitude's which both radial and low-order low-degree non-radial seismic waves
are predicted to have in such low mass stars \citep{Rodriguez-Lopez2019}. The
Jao Gap therefore provides one of the only current methods to probe the
interior physics of M dwarfs.

Stellar modeling has been successful in reproducing the Jao Gap
\citep[e.g.][]{Feiden2021,Mansfield2021} and, with these models, we have begun
to understand which parameters constrain the Jao Gap's location. For example,
it is now well documented that metallicity affects the Jao Gap's color, with
higher metallicity stellar populations showing the Jao Gap at consistently
higher masses / bluer colors \citep{Mansfield2021}.

% Both \citeauthor{Feiden2021} and \citeauthor{Mansfield2021} demonstrate the Jao
% Gap's location sensitivity to age, evolving to higher mass regions of the
% mass-luminosity relation with population age. Per \citet{Mansfield2021} the
% degree of this location evolution also does not seem to be strongly sensitive
% to metallicity. 


\input{src/sections/JaoGap.tex}

\section{Updated Opacities}\label{sec:opac}
% Radiative opacity is fundamental to stellar structure, it determines how much
% incident radiation is absorbed or scattered. Moreover, when a media is in
% thermodynamic equilibrium with the radiation field, that is when the temperature
% of the media and that of the radiation field is the same, the opacity may be
% used via Kirchhoff's law to find the emissivity of a material
% \citep{Huebner2014}. Local Thermodynamic Equilibrium (LTE) is a common state to
% find within a star and therefore stellar models have long relied on opacities
% calculated in LTE.
OPAL high-temperature radiative opacity tables in particular are very widely
used by current generation isochrone grids \citep[e.g. Dartmouth, MIST, \&
StarEvol, ][]{Dotter2008,Choi2016,Amard2019}. However, there are two primary
issues with these tables, one, they are relativly old and therefore do not
incorperate the most up to date understanding of plasma modeling in their code
{\color{red} [CITATION]}, and two, they report rossland mean opacitieis to only
{\color{red} N} digits {\color{red} [WHICH IS AN ISSUE WHY?]}.

While the two issues given above should have relativly small affects, the
strong theoretical opacity dependence of the Jao Gap raises the potential for
these small effects to measurably shift the gap's location. In order to address
both the out of date plasma modeling and the low numeric presicion we update
DSEP to use high temperature opacity tables based on measurements from Los
Alamos national Labs T-1 group \citep[OPLIB,][]{Colgan2016}. The OPLIB tables
use the much more up-to-date ATOMIC plasma modeling code {\color{red}
[CITATION]} in addition to reporting rossland mean opacities to {\color{red} N}
digits of numeric presicion.

ATOMIC \citep{Magee2004} is a LTE and non-LTE opacity and plasma modeling code.
A major strength of ATOMIC when compared to the older plasma modeling programs
is its ability to vary its refinement level \citep{Fontes2016}.
{\color{red}[OTHER DIFFERENCES]} For a more detailed breakdown of how the most
up-to-date set of OPLIB tables are generated see \citep{Colgan2016}.

The most up to date OPLIB tables include monochromatic Rosseland mean opacities
--- composed from bound-bound, bound-free, free-free, and scattering opacities
--- for elements hydrogen through zinc over temperatures 0.5eV to 100 keV and
for mass densities from approximately $10^{-8}$ g cm$^{-3}$ up to approximately
$10^{4}$ g cm$^{-3}$ (though the exact mass density range varies as a function
of temperature). 

When comparing OPAL and OPLIB opacity tables (Figure \ref{fig:opacComp}) we
find OPLIB opacities are systematically lower than OPAL opacities for
temperature above $10^{6}$ K (Figure \ref{fig:opacComp}). These lower opacities
will result the temperature where radiation dominates energy transport lowering
{\color{red} [CITATION?]}. Consequently, the radiative layer in stellar models
evolved using OPLIB opacity tables will be {\color{red} [further?]} out from
the models center than it would be in models making use of OPAL tables.

\begin{figure}
	\centering
	\includegraphics[width=0.45\textwidth]{src/figures/OpacityComparision.pdf}
	\caption{Rosseland mean opacity with the GS98 solar composition for both
	OPAL opacities and OPLIB opacities (top). Residuals between OPLIB opacities
	and OPAL opacities (bottom). These opacities are plotted at $\log _{10}(R)
	= -1.5$, $X=0.7$, and $Z=0.02$. Note how the OPLIB opacities are
	systematically lower than the OPAL opacities for temperatures above $10^6$
	K.}
	\label{fig:opacComp}
\end{figure}

\subsection{Table Querying and Conversion}
DSEP, along with most other stellar evolution programs, uses pre-computed
high-temperature opacity tables. Specifically, these tables list the
Rosseland-mean opacity, $\kappa_{R}$, along three dimensions: temperature, a
density proxy $R$, and composition. $R$ is defined as

\begin{align} \label{eqn:Req}
	R = \frac{\rho}{T_{6}^{3}}
\end{align}

Where $T_{6} = T\times10^{-6}$ and $\rho$ is the mass density. If $T$ and
$\rho$ are given in cgs then for much of the radius of a star $\log(R)\sim-1.5$
{\color{red}[CITATION]}.  $R$ is used, as opposed to simply tracking opacity
over mass density, because of its small dynamic range when compared to $\rho$ ($\rho\sim
10^{5}$ [g cm$^{-3}$] at the core of an RGB star all the way down to $\sim
10^{-8}$ [g cm$^{-3}$] within the envelope). 

OPLIB tables are queried from a web
interface\footnote{https://aphysics2.lanl.gov/apps/}. In order to generate many
tables easily and quickly we develope a web scraper built with Python's
\texttt{requests} module in addition to the 3rd party \texttt{mechanize} and
\texttt{BeautifulSoup} modules \citep{chandra2015python,
richardson2007beautiful} which can get tables with minimal intervention. This
web scraper submits a user requested chemical composition (composed of mass
fractions for elements from Hydrogen to Zinc) to the Los Alamos web form,
selects 0.0005 keV as the lower temperature bound and 60 keV as the upper
temperature bound, and finally requests opacity measurements for 100 densities,
ranging from $1.77827941\times 10 ^{-15}$ [g cm$^{-3}$] up to $1\times10^{7}$
[g cm$^{-3}$], at each temperature interval. These correspond to approximately
the same temperature and density range of opacities present in the OPAL opacity
tables.

OPLIB reports $\kappa_{R}$ as a function of mass density, temperature in keV,
and composition. Recall that DSEP accepts tables where opacity is given as a
function of temperature in Kelvin, $R$, and composition. The conversion from
temperature in keV to Kelvin is trivial
\begin{align}
	T_{K} = T_{keV} * 11604525.0061657
\end{align}
However, the conversion from mass density to $R$ is more involved. Because $R$ is
coupled with both mass density and temperature there there is no way to
directly convert tabulated values of opacity reported in the OPLIB tables to
their equivalents in $R$ space. Instead we must rotate the tables,
interpolating $\kappa_{R}(\rho,T_{eff}) \rightarrow \kappa_{R}(R,T_{eff})$. 

To preform this rotation we use the \texttt{interp2d} function within
\texttt{scipy}'s \texttt{interpolate} \citep{2020SciPy-NMeth} module to
construct a cubic bivariate B-spline \citep{Dierckx1981} interpolating function
$s$, with a smoothing factor of 0, representing the surface $\kappa_{R}(\rho,
T_{eff})$. For each $R^{i}$ and $T^{j}_{eff}$ which DSEP expects
high-temperature opacities to be reported for, we evaluate Equation
\ref{eqn:Req} to find $\rho^{ij} = \rho(T^{j}_{eff},R^{i})$.  Opacities in
$T_{eff}$, $R$ space are then inferred as $\kappa^{ij}_{R}(R^{i},T^{j}_{eff}) =
s(\rho^{ij}, T^{j}_{eff})$. 

As first-order validation of this interpolation scheme we can preform a similar
interpolation in the opposite direction, rotating the tables back to
$\kappa_{R}(\rho, T_{eff})$ and then comparing the initial, ``raw'', opacities
to those which have gone through the interpolations process. Figure
\ref{fig:fracdiff} shows the fractional difference between the raw opacities
and a set which have gone through this double interpolation. The red line
denotes $Log(R)=-1.5$ where models will tend to sit for much of their radius.
Along the $Log(R)=-1.5$ line the mean fractional difference is $\langle \delta
\rangle = 0.006$ with an uncertainty of $\sigma_{\langle\delta\rangle} =
0.009$. One point of note is that, because the initial rotation into $Log(R)$
space also reduces the domain of the opacity function interpolation-edge
effects which we avoid initially by extending the domain past what DSEP needs
cannot be avoided when interpolating back into $\rho$ space. {\color{red} [IS
THERE SOME MORE CLEAR VALIDATION WHICH I SHOULD PREFORM?]}

\begin{figure}
	\centering
	\includegraphics[width=0.45\textwidth]{src/figures/FractionalDifference.pdf}
	\caption{Log Fractional Difference between opacities in $\kappa_{R}(\rho,
	T_{eff})$ space directly queried from the OPLIB webform and those which
	have been interpolated into $Log(R)$ space and back. Note that, due to the
	temperature grid DSEP uses not aligning perfectly which the temperature
	grid OPLIB uses there may be edge effects where the interpolation is poorly
	constrained. The red line corresponds to $Log(R) = -1.5$ where much of a
	stellar model's radius exists.}
	\label{fig:fracdiff}
\end{figure}



\section{Modeling}\label{sec:modeling}
In order to address the two main issues with using OPAL opacity tables we use
our OPLIB opacity table web scraper to generate a set of tables that
consistently model lower metallicities. Specifically, we generate tables for
$Z_{\odot}=0.017$, $Z=0.01$, $Z=0.001$, and $Z=0.0001$. Compositions are
derived from the GS98 solar composition, with the mass fractions between metals
remaining constant, and only the total metal mass fraction is allowed to vary.
Moreover, Helium mass fraction is held constant as extra mass from the reduced
metallicity is put into additional Hydrogen. 

For each metallicity 101, uniformly spaced, models from 0.3 to 0.5 M$_{\odot}$
(spacing of 0.001 M$_{\odot}$) are evolve with both the GS98 OPAL opacity table
and OPLIB tables, hereafter these are the ``coarse'' models. For each set of
coarse models the discontinuity in the mass-luminosity relation is identified
at an age of 7 Gyr (Figures \ref{fig:coarseMassLum} \& \ref{fig:coarseTeffLum}
shows a characteristic example).

Immediately, the difference in mass where the discontinuity manifests is clear.
For each metallicity the discontinuity in the OPLIB models is approximately one
one-hundredth of a solar mass lower than the discontinuity in the OPAL models. We can
validate that this discontinuity is indeed correlated with the convective
transition mass; Figure \ref{fig:convTransition} shows an example of the model
forming radiative zones at approximately the same masses where the discontinuity
in the mass-luminosity function exists.

At this resolution only a few models exist within the
mass range of the discontinuity. In order to better constrain its location we
run a series of ``fine'' models, with a mass step of 0.0001 M$_{\odot}$ and
ranging from where the mass derivative first exceeds two sigma away from the
mean derivative value up to the mass where it last exceeds two sigma away from
the mean. A characteristic fine mass-luminosity relation is shown in Figure
\ref{fig:fineMassLum}.


\begin{figure}
	\centering
	\vspace{5mm}
	\includegraphics[width=0.45\textwidth]{src/figures/ConvectiveMassFraction.pdf}
	\caption{Convective Mass Fraction vs. initial model mass for Z=0.01 at 7
	Gyr (top), Derivative of luminosity with respect to mass for the OPAL and
	OPLIB models (bottom).  Note how the model transitions from fully
	convective at approximately the same mass where the discontinuity exists.}
	\label{fig:convTransition}
\end{figure}


\begin{figure}
	\centering
	\vspace{5mm}
	\includegraphics[width=0.45\textwidth]{src/figures/MassLumDisconZ001Paper-fine.pdf}
	\caption{Mass-Luminosity relation for Z=0.01 at 7 Gyr for models run with
	both OPAL and OPLIB high-temperature opacity tables and a mass step between
	them of 0.0001 M$_{\odot}$ (top). Derivative of luminosity with respect to
	mass for the OPAL and OPLIB models (bottom).}
	\label{fig:fineMassLum}
\end{figure}

Using the fine models we identify the location of the discontinuity in the same
manner as before, results of this are presented in Table
\ref{tab:fineMassRange}. Of note with the mass ranges we measure for the
discontinuity is that are generally not in agreement with those measured in
\citet{Mansfield2021}. However, the luminosity difference from over the gap
($\approx 0.1 mag$) is similar to both the observational difference and that
reported in \citet{Mansfield2021}. Currently, it is not clear why our mass
range is not in agreement with the \citet{Mansfield2021} mass range and further
investigation is therefore needed.


\section{Results}\label{sec:results}
We quantify the Jao Gap location along the magnitude (Table
\ref{tab:GapLocation}) axis by sub-sampling our synthetic populations, finding
the linear number density along the magnitude axis of each sub-sample,
averaging these linear number densities, and extracting any peaks above a
prominence threshold of 0.1 as potential magnitudes of the Jao Gap (Figure
\ref{fig:JaoGapLocator}). Gap widths are measuredat 50\% the height of the peak
prominence. We use the python package \texttt{scipy} \citep{2020SciPy-NMeth} to
both identify peaks and measure their widths. 

\begin{table}
	\centering
	\begin{tabular}{c | c c c}
		\hline
		Model & Location & Prominence & Width\\
		\hline
		\hline
		OPAL 1 & 10.138 & 0.593 & 0.027 \\
		OPAL 2 & 10.183 & 0.529 & 0.023 \\
		OPLIB 1 & 10.188 & 0.724 & 0.032 \\
		OPLIB 2 & 10.233 & 0.386 & 0.027 
	\end{tabular}
	\caption{Locations identified as potential Gaps.}
	\label{tab:GapLocation}
\end{table}

In both OPAL and OPLIB synthetic populations our Gap identification method
finds two gaps above the prominence threshold. The identification of more than
one gap is not inconsistent with the mass-luminosity relation seen in the grids
we evolve. As noise is injected into a synthetic population smaller features will
be smeared out while larger ones will tend to persist. The mass-luminosity
relations showin in Figure \ref{fig:PunchIn} make it clear that there are: (1),
multiple gaps due to stars of different masses undergoing convective mixing
events at different ages, and (2), the gaps decrease in width moving to lower
masses / redder. Therefore, the multiple gaps we identify are attributable to
the two bluest gaps being wide enough to not smear out with noise. In fact, if
we lower the prominence threshold just slightly from 0.1 to 0.09 we detect a
third gap in both the OPAL and OPLIB datasets where one would be expected.

\begin{figure}
	\centering
	\includegraphics[width=0.45\textwidth]{src/figures/NotebookFigs/OPAL_Jao_locator.pdf}
	\includegraphics[width=0.45\textwidth]{src/figures/NotebookFigs/OPLIB_Jao_locator.pdf}
	\caption{(right panels) OPAL (top) and OPLIB (bottom) synthetic
	populations. (left panels) Normalized linear number density along the
	magnitude axis. A dashed line has been extended from the peak through both
	panels to make clear where the identified Jao Gap location is wrt. to the
	population. }
	\label{fig:JaoGapLocator}
\end{figure}

The mean gap location of the OPLIB population is at a faiter magnitude than the
mean gap location of the OPAL population. Consequently, in the OPLIB sample the
convective mixing events which drive the kissing instability happen more
regularly and therefore also start earlier in the model's evolution. This is
because each mixing event serves to interrupt the ``standard'' luminosity
evolution of a stellar model, kicking its luminosity back down to what it would
have been at some earlier stage of stellar evolution instead of allowing it to
slowly increase.
% Looking at the interior physics of one OPAL and one OPLIB
% model shows that this shorter duration between mixing events (Figure
% \ref{fig:OPALOPLIB3He}).


Convective mixing events starting earlier in a model's evolution are consistent
with the slightly lower opacities characteristic to OPLIB. A lower opacity
fluid will have a more shallow radiative temperature gradient than a higher
opacity fluid; however, as the adiabatic temperature gradient remains
essentially unchanged as a function of radius, a larger interior radius of the
model will remain unstable to convection {\color{red}[CHECK IF THIS OR IF
RADIATIVE ZONE MOVING IN]}. This larger convective zone, and therefore smaller
radiative zone, is in line with the behavior of the models presented here as it
with the radiative zone closer to the convective zone it takes less time for
that radiative zone to heat up and become unstable to convection. We see that
OPLIB models undergo convective mixing events earlier in their evolution than
OPAL models (Figure \ref{fig:OPALOPLIB3He}) implying that the inner convective
zone did not have to expand as much to meet the outer convective zone. 


The most precise published Gap location comes from \citet{Jao2020} who use EDR3
to locate the Gap at $M_{G} \sim 10.3$, we identify the Gap at a similar
location in the GCNS data. \textbf{The Gap in populations evolved using OPLIB tables
is closer to this measurement than it is in populations evolved using OPAL tables
(Table \ref{tab:GapLocation}).} It should be noted that the exact location of
the observed Gap is poorly captured by a single value as the Gap visibly
compresses across the width of the main-sequence, wider on the blue edge and
narrower on the red edge such that the observed Gap has downward facing a wedge
shape (Figure \ref{fig:JaoGap}). This wedge shape is not successfully
reproduced by either any current models or the modeling we preform here. We
elect then to specify the Gap location where this wedge is at its narrowest, on
the red edge of the main sequence.

\begin{figure}
	\centering
	\includegraphics[width=0.5\textwidth]{src/figures/NotebookFigs/3HeOPAL_OPLIB.pdf}
	\caption{Core $^{3}$He mass fraction for a model evolved with OPAL and a
	model evolved with OPLIB within the Jao Gap's mass range. Note how the
	OPLIB model undergoes the mixing event earlier in its evolution than the
	OPAL model does.}
	\label{fig:OPALOPLIB3He}
\end{figure}

The Gaps identified in our modeling have widths of approximately 0.03
magnitudes, while the shift from OPAL to OPLIB opacities is 0.05 magnitudes.
With the prior that the Gaps clearly shift before noise is injected we know
that this shift is real. However, since the shift magnitude and Gap width are
of approximately the same size in our synthetic populations its likely that in
a real population --- with both compositional and age variations which we do
not account for --- \textbf{the Gap location will not provide a usable
constraint on the opacity source.}


\section{Conclusion}\label{sec:conclusion}
The Jao Gap provides an intriguing probe into the interior physics of M Dwarfs
stars where traditional methods of studying interiors break down. However,
before detailed physics may be inferred it is essential to have models which
are well matched to observations. Here we investigate whether the OPLIB opacity
tables reproduce the Jao Gap location and structure more accurately than the
widely used OPAL opacity tables. We find that while the OPLIB tables do shift
the Jao Gap location more in line with observations the shift is small enough
that it is likely not distinguishable from noise due to population age and
chemical variation. Moreover, we do not find that the OPLIB opacity tables help
in reproducing the wedge shape of the observed Gap.


\bibliography{src/bib/ms}{}
\bibliographystyle{aasjournal}


\end{document}

