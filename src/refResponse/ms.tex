% LaTeX rebuttal letter example. 
% 
% Copyright 2019 Friedemann Zenke, zenkelab.org
%
% Based on examples by Dirk Eddelbuettel, Fran and others from 
% https://tex.stackexchange.com/questions/2317/latex-style-or-macro-for-detailed-response-to-referee-report
% 
% Licensed under cc by-sa 3.0 with attribution required.

\documentclass[11pt]{article}
\usepackage[utf8]{inputenc}
\usepackage{lipsum} % to generate some filler text
\usepackage{fullpage}

% import Eq and Section references from the main manuscript where needed
% \usepackage{xr}
% \externaldocument{manuscript}

% package needed for optional arguments
\usepackage{xifthen}
% define counters for reviewers and their points
\newcounter{reviewer}
\setcounter{reviewer}{0}
\newcounter{point}[reviewer]
\setcounter{point}{0}

% This refines the format of how the reviewer/point reference will appear.
\renewcommand{\thepoint}{\arabic{point})} 

% command declarations for reviewer points and our responses
\newcommand{\reviewersection}{\stepcounter{reviewer} \bigskip \hrule
                  \section*{Reviewer \thereviewer}}

\newenvironment{point}
   {\refstepcounter{point} \bigskip \noindent {\textbf{Referee~Point~\thepoint} } ---\ }
   {\par }

\newcommand{\shortpoint}[1]{\refstepcounter{point}  \bigskip \noindent 
	{\textbf{Reviewer~Point~\thepoint} } ---~#1\par }

\newenvironment{reply}
   {\medskip \noindent \begin{sf}\textbf{Reply}:\  }
   {\medskip \end{sf}}

\newcommand{\shortreply}[2][]{\medskip \noindent \begin{sf}\textbf{Reply}:\  #2
	\ifthenelse{\equal{#1}{}}{}{ \hfill \footnotesize (#1)}%
	\medskip \end{sf}}

\begin{document}

\section*{Response to the referee}
% General intro text goes here
We thank the referee for their critical assessment of our work. 
In the following we address their concerns point by point. 

% Let's start point-by-point with Reviewer 1
\hrule
% Point one description 
\begin{point}
	Since neither set of models using the two different opacity tables provide a
	good match to the properties of the Jao gap, the possibility that the
	convective kissing instability is not the correct explanation for the gap
	has to be considered.

	The first paper to offer an explanation for the existence of the gap (MacDonald
	\& Gizis 2018) states 'Convective mixing is treated as a diffusion process
	with the diffusion coefficient determined from mixing length theory.' and
	'The fully implicit nature of our code also prevents the convective kissing
	instability discovered and described by van Saders \& Pinsonneault (2012).'
	The treatment of convective mixing as a diffusion process is probably the
	reason why MacDonald \& Gizis find a single episode of convection zone
	merger (for models in which merger occurs) and hence a single dip in the
	luminosity function, which seems to be the case for the Jao gap (inferred
	from figure 1 of this paper).

	In contrast, the use of instantaneous mixing leads to multiple episodes of
	convection zone merger and this could be the reason why the authors find
	two dips in the luminosity function (inferred from their figures 7 and 8).
	The authors need to give in the paper a convincing argument as to why the
	instantaneous mixing approximation is valid, particularly as using the
	diffusion approach seems to more physically consistent with current
	understanding of turbulent mixing. This should involve estimates using
	mixing length theory of the mixing time scale at all stages of the merger,
	with special attention to mixing time scales within one mixing length on
	either side of the point of contact between the merging convection zones.
	This region is where the mixing time scale is most likely to be the
	longest.

	The mixing time scale then needs to compared to other relevant time scales
	including the time scale for deuterium to come into equilibrium and the
	time scale at which the helium-3 abundance is modified by nuclear
	reactions.

	If it turns out that the instantaneous mixing approximation is not valid, then
	there needs to be discussion of alternative approaches and their
	consequences. The thrust of the paper could be changed to show that the CKI
	is not the correct explanation of the gap.
\label{pt:timescale}
\end{point}

% Our reply
\begin{reply}
	We disagree with the referee that the a diffusive model of convective
	mixing should be prefered over an instantaneous model of convective mixing.
	Much of the following has been included as text in the manuscript.

	As the referee says the primary point to clarify here is the ratio of the
	overturn timescale to the timestep length. DSEP treats convective mixing
	instantanously throught a single shell and we evolve models composed of
	5000 radial shells. Convective overturn timescales for M-dwarfs have not
	insignifigant uncertanties on them (with different sources reporting
	overturn times between 70 and 300 days). However, all of these estimates
	are order of magnitude shorter than the time steps used (which bottom out
	at around 1 Myr). Therefore, convective mixing is extremly well
	approximated by instantanious mixing.

	To get a better sense of this for our models we find the overturn time for
	a single shell at the radial depth where convective kissing instability
	happens in our models and compare this to our time step length. For a
	single shell the overturn time is [DAYS], which is [FRACTION] shorter than
	the time steps we used.

	Moreover, through private communication with Gregory Feiden, a leading
	expert on the Jao Gap, we have recived independent cooberation of these
	time scales.

	Therefore, we do belive that instanious mixing is a valid appoximation to
	make for these models. The thrust of this paper will therefore remain the
	same.

	The text addressing this has been added into the section entitled Modeling.
\end{reply}

\begin{point}
	The convective mixing method used by MacDonald \& Gizis needs to be properly
	described along with the resulting differences from using instantaneous
	mixing.
	\label{pt:address}
\end{point}

\begin{reply}
	Additional text has been added to the introduction of the paper addressing the
	resulst of MacDonald \& Gizis. 
\end{reply}

\begin{point}
	On line 184, the authors say that the OPLIB tables were created to resolve the
	discrepancy between helioseismic and solar model predictions of chemical
	abundances in the Sun. I find this to be an odd way to phrase the problem.
	Presumably the authors are referring to the difficulty of making solar
	models that match the sound speed profile (or almost equivalently the depth
	of the surface convection zone) determined by helioseismology given the
	composition constraints provided by the then recent new measurements of the
	surface abundances, notably the oxygen abundance. The authors should
	rephrase the problem and also discuss whether using OPLIB opacities help
	resolve the problem or not.
	\label{pt:seismo}
\end{point}

\begin{reply}
	We agree with the refee that the invocation of helioseismic discrepencies
	is confusing. Because this is not a helioseismic paper we drop from the
	text, instead just mentioning that the OPLIB tables make use of the most up
	to date physics, which is the most relevant point to our work. We do note
	here though that the OPLIB opacities did not serve to resolve any
	discrepencies.
\end{reply}

\begin{point}
	The distinction between low temperature and high temperature opacity sources
	made in the paragraph beginning on line 196 is somewhat artificial.
	Presumably the transition between low temperature and high temperature
	opacity is chosen to be between $10^{4.3}$ and $10^{4.5}$ K because there
	is where the Ferguson et al. opacities and OPAL opacities are close to each
	other. Perhaps the evolution could be affected if a different temperature
	range for the transition is used. The authors should stress in this section
	that as far as modeling the gap the main impact of using different
	opacities is on the radiative zone, and give the temperature and R (or
	density) ranges that are relevant to the radiative zone(s) so that the
	reader can see from figure 3 the expected change in opacity, and also if
	log R = -1.5 is truly a representative value in the radiative zone.

	Also, from figure 3, it seems that the OPLIB opacities are lower than the OPAL
	opacities for temperatures greater than $10^{5.5}$ K and not $10^{5}$ K as
	stated in the figure caption and also on line 203.
	\label{pt:lowtempboundary}
\end{point}

\begin{reply}
	The range where DSEP ramps from low temperature to high temperature
	opacities is determined by the temperature range where molecules can start
	to form. It is important for the low temperature opacities to be the only
	opacity source by the time the first molecules start forming. We choose to
	ramp the opacity source so that there is not a hard discontinuity. This
	same ramp has been used as standard in all DSEP models since 2008, see
	Dotter et al. 2008 for further details.
\end{reply}

\begin{point}
	In section 3.2, mention is made of the solar surface Z/X ratio but the actual
	value is not given. Is it the value recently determined by Magg et al.
	(2022), Z/X = 0.0225 or some other earlier value? The authors should state
	the actual Z/X value used.

	Also, the authors need to say whether or not they include gravitational
	settling and element diffusion in their solar modeling, and if they do, say
	how it is done (e.g. are elements grouped or treated independently). The
	authors should also include discussion of how well their solar models
	replicate the sound speed profile determined from helioseismology.
	\label{pt:ZX}
\end{point}

\begin{reply}
	The calibrated Z/X value has been added into the text in section 3.2 along
	with clarification that we do include gravitational setteling with elements
	grouped together. We do not include a disscusion of sound speed in our
	manuscript. This paper is not a disscusion of seismology so a diversion to that
	would be disctracting for readers and out of place.
\end{reply}

\begin{point}
	Presumably, the authors use their solar calibrated models to set the mixing
	length ratio and initial abundances for their calculations of the evolution
	of models of low mass stars. Do they use primordial or present-day solar
	abundances? Why should the mixing length ratio be the same as the solar
	calibrated? There is evidence that the mixing length varies with stellar
	properties (e.g. Trampedach et al. 2014; Joyce \& Chaboyer 2018). A better
	fit to the location of the gap might be obtained by adjusting the mixing
	length ratio. The authors need to address these questions.
	\label{pt:alpha}
\end{point}

\begin{reply}
	We use GS98 solar abundances [CITATION] for all models. While, as the
	referee says, there is substanntial evidence of a metallicity dependence
	for the mixing length parameter this dependence has only been shown to have
	a substantial effect on higher mass stars. However, to fully address this
	point we have run an additional grid of models with the mixing length
	parameter dramatically lowered ($\alpha_{ML} = 1.5$). Results of that grid
	are shown in Figure [FIGURE]. Of primary note however, is that the Jao Gap
	loaction is within one sigma of where we detect in with our solar
	calibrated mixing length. Text has been added into the paper clarifying
	that we use a solar calibrated mixing length but that the actual Jao Gap
	location of our models is only very weekly sensitive to mixing length.
\end{reply}

\begin{point}
	On trying the web interface for the OPLIB, it seems that the process of
	interpolating between rho and R is unnecessary. Once T is specified, it is
	possible to get the same set of R values as for the OPAL tables by
	specifying the starting value of rho. Then only interpolation in T is
	needed to get a table in the same form as the OPAL tables. The authors need
	to clarify why they chose their approach. Also, it would be helpful to
	include in figure 15 a line plot of fractional difference against log T for
	log R = -1.5 or a different R value if -1.5 is not found to be
	representative for the radiative zone (see point 4). \label{pt:webR}
\end{point}

\begin{reply}
	While it is possible to pick out particularl $R$ values and get them
	directly with the OPLIB webform it is not possible to get the same grid of
	R values from the web form as the type 2 OPAL tables report opacities over.
	This is because the grid of temperatures OPLIB populates does not line up
	with what would be needed. Therefore, the interpolation scheme described in
	Appendix B is required.
\end{reply}

\subsection*{Minor}

% Use the short-hand macros for one-liners.
\shortpoint{ Typo in line xy. }
\shortreply{ Fixed.}


\end{document}
