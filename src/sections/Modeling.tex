\section{Modeling}\label{sec:modeling}
In order to model the Jao Gap we evolve two extremely finely sampled mass grids
of models. One of these grids uses the OPAL high-temperature opacity tables
while the other uses the OPLIB tables (Figure \ref{fig:PunchIn}). Each grid
evolves a model every 0.00025 $M_{\odot}$ from 0.2 to 0.4 $M_{\odot}$ and every
0.005 $M_{\odot}$ from 0.4 to 0.8 $M_{\odot}$. All models in both grids use a
GS98 solar composition, the (1, 101, 0) \texttt{Free\_EOS} (version
{\color{red}2.7}) configuration, and 1000 year old pre-main sequence polytropic
models, with polytropic index 1.5, as their initial conditions.

Because in this work we are just interested in the location shift of the Gap as
the opacity source varies, we do not model variations in composition.
\citet{Mansfield2021,Jao2020,Feiden2021} all look at the effect composition has
on Jao Gap location. They find that as population metallicity increases so too
does the mass range and consequently the magnitude of the Gap. From an extremely
low metallicity population (Z=0.001) to a population with a more solar like
metallicity this shift in mass range can be up to 0.05 M$_{\odot}$
\citep{Mansfield2021}.

\begin{figure}
	\centering
	\includegraphics[width=0.45\textwidth]{src/figures/NotebookFigs/OPALPunchIn.pdf}
	\includegraphics[width=0.45\textwidth]{src/figures/NotebookFigs/OPLIBPunchIn.pdf}
	\caption{Mass-luminosity relation at 7 Gyrs for models evolved using OPAL opacity
	tables (top) and those evolved using OPLIB opacity tables (bottom). Note
	the lower mass range of the OPLIB Gap.}
	\label{fig:PunchIn}
		
\end{figure}

\subsection{Population Synthesis}
In order to compare the Gap to observations we use in house population
synthesis code. We empirically calibrate the relation between G, BP, and RP
magnitudes and their uncertainties along with the parallax/G magnitude
uncertainty relation using the GCNS and Equations \ref{eqn:plxCalib} \&
\ref{eqn:MagCalib}. The full series of steps in our population synthesis code
are

\begin{align}\label{eqn:plxCalib}
	\sigma_{plx}(M_{g}) = ae^{bM_{g}}+c
\end{align}
\begin{align}\label{eqn:MagCalib}
	\sigma_{i}(M_{i}) = ae^{M_{i}-b}+c
\end{align}

\begin{figure}
	\centering
	\includegraphics[width=0.45\textwidth]{src/figures/NotebookFigs/pdist.pdf}
	\caption{Probability distribution sampled when assigning true parallaxes to
	synthetic stars. This distribution is built from the GCNS and includes all
	stars with BP-RP colors between 2.3 and 2.9, the same color range
	of the Jao Gap.}
	\label{fig:pdist}
\end{figure}

\begin{enumerate}
	\item Sample from a \citet{Sollima2019} ($0.25 M_{\odot} < M < 1 M_{\odot}$,
		$\alpha=-1.34\pm0.07$) IMF to determine synthetic star mass.
	\item Find the closest model above and below the synthetic star, lineally
		interpolate these models' $T_{eff}$, $\log(g)$, and $\log(L)$ to those
		at the synthetic star mass.
	\item Convert synthetic star $g$, $T_{eff}$, and $Log(L)$ to Gaia G, BP,
		and RP magnitudes using the Gaia (E)DR3 bolometric corrections
		\citep{Creevey2022} along with code obtained thorough personal
		communication with Aaron Dotter \citep{Choi2016}.
	\item Sample from the GCNS, limited to the BP-RP color range of
		2.3 -- 2.9, to assign synthetic star a ``true'' parallax.
	\item Use the true parallax to find an apparent magnitude for each filter.
	\item Evaluate the empirical calibration given in Equation
		\ref{eqn:plxCalib} to find an associated parallax uncertainty and
		adjust the true parallax by this value resulting in an ``observed''
		parallax.
	\item Use the ``observed'' parallax and the apparent magnitude to find an
		``observed'' magnitude.
	\item Fit the empirical calibration given in Equation \ref{eqn:MagCalib} to
		the GCNS and evaluate it to give a magnitude uncertainty scale in each
		band.
	\item Adjust each magnitude by an amount sampled from a normal
		distribution with a standard deviation of the magnitude uncertainty
		scale found in the previous step.
\end{enumerate}

This method then incorporates both photometric and astrometric uncertainties
into our population synthesis. An example 7 Gyr old synthetic populations
using OPAL and OPLIB opacities are presented in Figure
\ref{fig:PopSynthCompareBasic}.

\begin{figure*}
	\centering
	\includegraphics[width=0.85\textwidth]{src/figures/NotebookFigs/OPALOPLIB_popsynth_compare.pdf}
	\caption{Population synthesis results for models evolved with OPAL (left)
	and models evolved with OPLIB (right). A Gaussian kernel-density estimate
	has been overlaid to better highlight the density variations.}
	\label{fig:PopSynthCompareBasic}
\end{figure*}
