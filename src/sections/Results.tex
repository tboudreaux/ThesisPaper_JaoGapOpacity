\section{Results}\label{sec:results}
We quantify the Jao Gap location along the magnitude (Table
\ref{tab:GapLocation}) axis by sub-sampling our synthetic populations, finding
the linear number density along the magnitude axis of each sub-sample,
averaging these linear number densities, and extracting any peaks above a
prominence threshold of 0.1 as potential magnitudes of the Jao Gap (Figure
\ref{fig:JaoGapLocator}). Gap widths are measuredat 50\% the height of the peak
prominence. We use the python package \texttt{scipy} \citep{2020SciPy-NMeth} to
both identify peaks and measure their widths. 

\begin{table}
	\centering
	\begin{tabular}{c | c c c}
		\hline
		Model & Location & Prominence & Width\\
		\hline
		\hline
		OPAL 1 & 10.138 & 0.593 & 0.027 \\
		OPAL 2 & 10.183 & 0.529 & 0.023 \\
		OPLIB 1 & 10.188 & 0.724 & 0.032 \\
		OPLIB 2 & 10.233 & 0.386 & 0.027 
	\end{tabular}
	\caption{Locations identified as potential Gaps.}
	\label{tab:GapLocation}
\end{table}

In both OPAL and OPLIB synthetic populations our Gap identification method
finds two gaps above the prominence threshold. The identification of more than
one gap is not inconsistent with the mass-luminosity relation seen in the grids
we evolve. As noise is injected into a synthetic population smaller features will
be smeared out while larger ones will tend to persist. The mass-luminosity
relations showin in Figure \ref{fig:PunchIn} make it clear that there are: (1),
multiple gaps due to stars of different masses undergoing convective mixing
events at different ages, and (2), the gaps decrease in width moving to lower
masses / redder. Therefore, the multiple gaps we identify are attributable to
the two bluest gaps being wide enough to not smear out with noise. In fact, if
we lower the prominence threshold just slightly from 0.1 to 0.09 we detect a
third gap in both the OPAL and OPLIB datasets where one would be expected.

\begin{figure}
	\centering
	\includegraphics[width=0.45\textwidth]{src/figures/NotebookFigs/OPAL_Jao_locator.pdf}
	\includegraphics[width=0.45\textwidth]{src/figures/NotebookFigs/OPLIB_Jao_locator.pdf}
	\caption{(right panels) OPAL (top) and OPLIB (bottom) synthetic
	populations. (left panels) Normalized linear number density along the
	magnitude axis. A dashed line has been extended from the peak through both
	panels to make clear where the identified Jao Gap location is wrt. to the
	population. }
	\label{fig:JaoGapLocator}
\end{figure}

The mean gap location of the OPLIB population is at a faiter magnitude than the
mean gap location of the OPAL population. Consequently, in the OPLIB sample the
convective mixing events which drive the kissing instability happen more
regularly and therefore also start earlier in the model's evolution. This is
because each mixing event serves to interrupt the ``standard'' luminosity
evolution of a stellar model, kicking its luminosity back down to what it would
have been at some earlier stage of stellar evolution instead of allowing it to
slowly increase.
% Looking at the interior physics of one OPAL and one OPLIB
% model shows that this shorter duration between mixing events (Figure
% \ref{fig:OPALOPLIB3He}).


Convective mixing events starting earlier in a model's evolution are consistent
with the slightly lower opacities characteristic to OPLIB. A lower opacity
fluid will have a more shallow radiative temperature gradient than a higher
opacity fluid; however, as the adiabatic temperature gradient remains
essentially unchanged as a function of radius, a larger interior radius of the
model will remain unstable to convection {\color{red}[CHECK IF THIS OR IF
RADIATIVE ZONE MOVING IN]}. This larger convective zone, and therefore smaller
radiative zone, is in line with the behavior of the models presented here as it
with the radiative zone closer to the convective zone it takes less time for
that radiative zone to heat up and become unstable to convection. We see that
OPLIB models undergo convective mixing events earlier in their evolution than
OPAL models (Figure \ref{fig:OPALOPLIB3He}) implying that the inner convective
zone did not have to expand as much to meet the outer convective zone. 


The most precise published Gap location comes from \citet{Jao2020} who use EDR3
to locate the Gap at $M_{G} \sim 10.3$, we identify the Gap at a similar
location in the GCNS data. \textbf{The Gap in populations evolved using OPLIB tables
is closer to this measurement than it is in populations evolved using OPAL tables
(Table \ref{tab:GapLocation}).} It should be noted that the exact location of
the observed Gap is poorly captured by a single value as the Gap visibly
compresses across the width of the main-sequence, wider on the blue edge and
narrower on the red edge such that the observed Gap has downward facing a wedge
shape (Figure \ref{fig:JaoGap}). This wedge shape is not successfully
reproduced by either any current models or the modeling we preform here. We
elect then to specify the Gap location where this wedge is at its narrowest, on
the red edge of the main sequence.

\begin{figure}
	\centering
	\includegraphics[width=0.5\textwidth]{src/figures/NotebookFigs/3HeOPAL_OPLIB.pdf}
	\caption{Core $^{3}$He mass fraction for a model evolved with OPAL and a
	model evolved with OPLIB within the Jao Gap's mass range. Note how the
	OPLIB model undergoes the mixing event earlier in its evolution than the
	OPAL model does.}
	\label{fig:OPALOPLIB3He}
\end{figure}

The Gaps identified in our modeling have widths of approximately 0.03
magnitudes, while the shift from OPAL to OPLIB opacities is 0.05 magnitudes.
With the prior that the Gaps clearly shift before noise is injected we know
that this shift is real. However, since the shift magnitude and Gap width are
of approximately the same size in our synthetic populations its likely that in
a real population --- with both compositional and age variations which we do
not account for --- \textbf{the Gap location will not provide a usable
constraint on the opacity source.}
