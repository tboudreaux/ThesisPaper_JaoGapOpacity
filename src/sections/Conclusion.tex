\section{Conclusion}\label{sec:conclusion}
The Jao Gap provides an intriguing probe into the interior physics of M Dwarfs
stars where traditional methods of studying interiors break down. However,
before detailed physics may be inferred it is essential to have models which
are well matched to observations. Here we investigate whether the OPLIB opacity
tables reproduce the Jao Gap location and structure more accurately than the
widely used OPAL opacity tables. We find that while the OPLIB tables do shift
the Jao Gap location more in line with observations, by approximately 0.05
magnitudes, the shift is small enough that it is likely not distinguishable
from noise due to population age and chemical variation. However, future
measurement of [Fe/H] for stars within the gap will be helpful in constraining
the degree to which the gap should be smeared by these theoretical models.
Finally, we do not find that the OPLIB opacity tables help in reproducing the
as yet unexplained wedge shape of the observed Gap.
