\section{Conclusion}\label{sec:conclusion}
The Jao Gap provides an intriguing probe into the interior physics of M Dwarfs
stars where traditional methods of studying interiors break down. However,
before detailed physics may be inferred it is essential to have models which
are well matched to observations. Here we investigate whether the OPLIB opacity
tables reproduce the Jao Gap location and structure more accurately than the
widely used OPAL opacity tables. We find that while the OPLIB tables do shift
the Jao Gap location more in line with observations the shift is small enough
that it is likely not distinguishable from noise due to population age and
chemical variation. Moreover, we do not find that the OPLIB opacity tables help
in reproducing the wedge shape of the observed Gap.
