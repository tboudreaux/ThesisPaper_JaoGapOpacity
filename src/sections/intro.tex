\section{INTRODUCTION}\label{sec:intro}
Due to the initial mass requirements of the molecular clouds which collapse to form
stars, star formation is strongly biased towards lower mass, later spectral
class, stars when compared to higher mass stars. Partly as a result of this
bias and partly as a result of their extremely long main-sequence lifetimes,
M-dwarfs make up approximately 70 percent of all stars in the galaxy. Moreover,
some planet search campaigns have focused on M-dwarfs due to the relative ease
of detecting small planets in their habitable zones \citep[e.g.][]{Nut08}.
M-dwarfs then represent both a key component of the galactic stellar population
as well as the possible set of stars which may host habitable exoplanets.
Given this key location M-dwarfs occupy in modern astronomy it is important to
have a thorough understanding of their structure and evolution.

\citet{Jao2018} discovered a novel feature in the Gaia DR2 $G_{BP}-G_{RP}$
color-magnitude-diagram. Around $M_{G}=10$ there is an approximately 17\%
decrease in stellar density of the sample of stars \citeauthor{Jao2018}
considered. Subsequently, this has become known as either the Jao Gap, or Gaia
M dwarf Gap. Following the initial detection of the Gap in DR2 the gap has also
potentially been observed in 2MASS \citep{Skrutskie2006, Jao2018}; however, the
significance of this detection is quite weak and it relies on the prior of the gaps
location from Gaia data. Further, both EDR3 data and DR3 data also reveal the
gap \citep{Jao2020} {\color{red} [Check on these citations and find a citation
for DR3]}. These three data sets sources provided a clear picture that this
feature is not a bias inherent to DR2.

The Gap is generally attributed to convective instabilities in the cores of
stars straddling the fully convective transition mass (0.3 - 0.35 M$_{\odot}$)
\citep{Baraffe2018}. These instabilities interrupt the normal, slow, main
sequence luminosity evolution of a star and resulting in luminosities lower
than expected from the main sequence mass-luminosity relation \citep{Jao2020}.

The Jao Gap, inherently a feature of M dwarf populations, provides an enticing
and unique view into the interior physics of these stars \citep{Feiden2021}.
This is especially important as, unlike more massive stars, M dwarf seismology
is infeasible due to the short periods and extremely small
magnitude's which both radial and low-order low-degree non-radial seismic waves
are predicted to have in such low mass stars \citep{Rodriguez-Lopez2019}. The
Jao Gap therefore provides one of the only current methods to probe the
interior physics of M dwarfs \citep[e.g][{\color{red}[ARE THESE THE BEST PAPERS
TO CITE HERE?]}]{Feiden2021, Mansfield2021}.

Despite the early success modeling the Gap some issues remain.
\citeauthor{Jao2020} identify that the gap has a wedge shape which has not been
successful reproduced by any current modeling efforts and which implies a
somewhat unusual population composition of young, metal-poor stars. Further,
\citet{Feiden2021} identify substructure, an additional over density of stars,
directly below the Gap, again a feature not yet fully captured by current
models. 

All currently published {\color{red} [pretty sure this is correct but DOUBLE
CHECK THAT IT IS ALL OF THEM]} models of the Jao Gap make use of OPAL high
temperature radiative opacities. Here we investigate the affect of using the
more up to date OPLIB high temperature radiative opacities and whether these
opacity tables bring models more in line with observations. In section
\ref{sec:JaoGap} we provide an overview of the physics believed to result in the
Jao Gap, in section \ref{sec:opac} we review the differences between OPAL
and OPLIB and describe how we update DSEP to use OPLIB opacity tables. In
section \ref{sec:SCSM} we validate the update opacities by generating solar
calibrated stellar models. Section \ref{sec:modeling} walks through the stellar
evolution and population synthesis modelling we preform and finally Section
\ref{sec:results} presents our findings. {\color{red} [Make this more active]}

% Stellar modeling has been successful in reproducing the Jao Gap
% \citep[e.g.][]{Feiden2021,Mansfield2021} and, with these models, we have begun
% to understand which parameters constrain the Jao Gap's location. For example,
% it is now well documented that metallicity affects the Jao Gap's color, with
% higher metallicity stellar populations showing the Jao Gap at consistently
% higher masses / bluer colors \citep{Mansfield2021}.
%
% {\color{red} EXPAND THIS, READ SOME OTHER GAP PAPERS TO SEE WHAT THEY DO}

% Both \citeauthor{Feiden2021} and \citeauthor{Mansfield2021} demonstrate the Jao
% Gap's location sensitivity to age, evolving to higher mass regions of the
% mass-luminosity relation with population age. Per \citet{Mansfield2021} the
% degree of this location evolution also does not seem to be strongly sensitive
% to metallicity. 
