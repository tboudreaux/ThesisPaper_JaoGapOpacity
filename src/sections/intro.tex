\section{INTRODUCTION}\label{sec:intro}
\citet{Jao2018} discovered a novel feature in the Gaia $G_{BP}-G_{RP}$
color-magnitude-diagram. Around $M_{G}=10$ there is an approximately 17\%
decrease in stellar density of the sample of stars \citeauthor{Jao2018}
considered. Subsequently, this has become known as either the Jao Gap, or Gaia
M dwarf Gap. Section \ref{sec:p1} will go into more detail regarding the
physics underpinning this feature; however, in brief convective instabilities
in the core are believed to form for stars straddling the fully convective
transition mass (0.3 - 0.35 M$_{\odot}$) \citep{Baraffe2018}. These
instabilities interupt the normal, slow, main sequence luminosity evolution of
a star and resulting in lower than expected luminosities {\color{red} [WORDING]} \citep{Jao2020}.

The Jao Gap, inherently a feature of M dwarf populations, provides an enticing
and unique view into the interior physics of these stars \citep{Feiden2021}.
This is especially important as, unlike more massive stars, M dwarf seismology
is currently infeasible due to the short periods and extremely small
magnitude's which both radial and low-order low-degree non-radial seismic waves
are predicted to have in such low mass stars \citep{Rodriguez-Lopez2019}. The
Jao Gap therefore provides one of the only current methods to probe the
interior physics of M dwarfs.

Stellar modeling has been successful in reproducing the Jao Gap
\citep[e.g.][]{Feiden2021,Mansfield2021} and, with these models, we have begun
to understand which parameters constrain the Jao Gap's location. For example,
it is now well documented that metallicity affects the Jao Gap's color, with
higher metallicity stellar populations showing the Jao Gap at consistently
higher masses / bluer colors \citep{Mansfield2021}.

{\color{red} EXPAND THIS, READ SOME OTHER GAP PAPERS TO SEE WHAT THEY DO}

% Both \citeauthor{Feiden2021} and \citeauthor{Mansfield2021} demonstrate the Jao
% Gap's location sensitivity to age, evolving to higher mass regions of the
% mass-luminosity relation with population age. Per \citet{Mansfield2021} the
% degree of this location evolution also does not seem to be strongly sensitive
% to metallicity. 
